\documentclass[twocolumn]{article}
\usepackage[margin=1in]{geometry}
\usepackage{outlines}
\usepackage{amsmath}
\usepackage{gensymb}
\title{Ch. 25 Notes}
\author{John Yang}
\setcounter{section}{+24}

\begin{document}
\maketitle
\section{The Special Theory of Relativity}
\subsection{Reference Frames}
\begin{outline}
\1 
\end{outline}
\subsection{Classical Gallilean Relativity}
\begin{outline}
\1 
\end{outline}
\subsection{The Need for Change and the Postulates of the Special Theory}
\begin{outline}
\1 
\end{outline}
\subsection{Time Dilation}
\begin{outline}
\1 
\end{outline}
\subsection{Lengths Perpendicular to the Direction of Motion}
\begin{outline}
\1 
\end{outline}
\subsection{Lengths Oriented Along the Direction of Motion: Length Contraction}
\begin{outline}
\1 
\end{outline}
\subsection{The Lorentz Transformation Equations}
\begin{outline}
\1 
\end{outline}
\subsection{The Relativity of Simultaneity}
\begin{outline}
\1 
\end{outline}
\subsection{A Relativistic Centipede}
\begin{outline}
\1 
\end{outline}
\subsection{A Relativistic Paradox and Its Resolution}
\begin{outline}
\1 
\end{outline}
\subsection{Relativistic Velocity Addition}
\begin{outline}
\1 
\end{outline}
\subsection{Cosmic Jets and the Optical Illusion of Superluminal Speeds}
\begin{outline}
\1 
\end{outline}
\subsection{The Longitudinal Doppler Effect}
\begin{outline}
\1 
\end{outline}
\subsection{The Transverse Doppler Effect}
\begin{outline}
\1 
\end{outline}
\subsection{A General Equation for the Relativistic Doppler Effect}
\begin{outline}
\1 
\end{outline}
\subsection{Relativistic Momentum}
\begin{outline}
\1 
\end{outline}
\subsection{The CWE Theorem Revisited}
\begin{outline}
\1 
\end{outline}
\subsection{Implications of the Equivalence Between Mass Energy}
\begin{outline}
\1 
\end{outline}
\subsection{Space-Time Diagrams}
\begin{outline}
\1 
\end{outline}
\end{document}
