\documentclass[twocolumn]{article}
\usepackage[margin=1in]{geometry}
\usepackage{outlines}
\usepackage{amsmath}
\usepackage{gensymb}
\title{Ch. 23-24 Notes}
\author{John Yang}
\setcounter{section}{+22}

\begin{document}
\maketitle
\section{Geometric Optics}
\subsection{The Domains of Optics}
\begin{outline}
\1 Assume that light travels in straight lines called rays
\1 domains of optics: 
    \2 Geometric optics: only rays 
    \2 Physical optics: rays and waves 
    \2 quantum (photon) optics: rays, waves, energy bundles called photons 
\1 If the wavelength is much less than the size of the opening or obstacle presented to the light, then the geometric limit is appropriate and each incident light ray subsequently travels in a unique direction or single ray. Well defined shadows exist in regions where there are no rays. This is the domain of geometric optics. 
\1 If the wavelength is on the order of the size of the opening through which the light passes, then the wave limit is appropriate and we must account for diffraction effects. THis is the domain of physical optics. 

\end{outline}
\subsection{The Inverse Square Law for Light}
\begin{outline}
\1 Intensity of light at a distance obeys the inverse square law. Pointlike sources of light emit rays equally in all directions. 
\1 Luminosity is the energy per second of the light source. \[I\equiv\dfrac{L}{4\pi r^2}\] where $r$ is the distance from the source, $I$ is the intensity, and $L$ is the luminosity. 
\1 Brightness is intensity times the area of the aperture; that is, \[B\equiv IA\]


\end{outline}
\subsection{The Law of Reflection}
\begin{outline}
\1 Reflected light obeys \[\theta=\theta'\] where the angles are the angles the rays make with the normal line, which is the line perpendicular to the surface. Assumptions: incident, normal line, and reflected ray are all in the same plane. 
\end{outline}
\subsection{The law of refraction}
\begin{outline}
\1 At the interface, a boundary line between two mediums, some light always reflects and some light refracts. 
\1 Law of refraction: \[n_1\sin\theta_1=n_2\sin\theta_2\], where $n$ is the index of refraction of a substance, which is greater or equal to 1. 
\end{outline}
\subsection{Total Internal Reflection}
\begin{outline}
\1 Total internal reflection occurs when the incidence angle exceeds the critical angle, \[\sin\theta_c=\dfrac{n_2}{n_1}\] where \(n_2>n_1\)
\end{outline}
\subsection{Dispersion}
\begin{outline}
\1 For some materials, the index of refraction varies with the wavelength of the light (dispersion), which means that the constituent colors of a mixed light separate when entering such a material. See: glass prism. 

\end{outline}
\subsection{Rainbows}
\begin{outline}
\1 Rainbows occur when light refracts through small water droplets in a mist. The rays experience dispersion, causing the colors to separate and the rainbow to appear. 

\end{outline}
\subsection{Optics and Images}
\begin{outline}
\1 If the existing light rays intersect, or even appear to intersect, at some point, that point is the image point of the object point. If these rays physically converge at the image point, the image is a real image (the rays really intersect there). If the exiting rays only appear to intersect at the image point, the image is a virtual image. 
    \2 For single lens and single mirror systems, upright is virtual and real is inverted. 

\end{outline}
\subsection{The Cartesian Sign Convention}
\begin{outline}
\1 The object is placed to the left of the optical device. The light travels initially from left to right. 
\1 The center of the optical device is the origin; called the vertex. The horizontal axis is the optic axis. 
\1 Magnification $m$ - ratio of size of image to size of object. Positive $m$ means upright; negative $m$ means inverted. 

\end{outline}
\subsection{Imagine Formations by Spherical and Plane Mirrors}
\begin{outline}
\1 
\end{outline}
\subsection{Ray Diagrams for Mirrors}
\begin{outline}
\1 
\end{outline}
\subsection{Refraction at a Single Spherical Surface}
\begin{outline}
\1 
\end{outline}
\subsection{Thin Lenses}
\begin{outline}
\1 
\end{outline}
\subsection{Ray Diagrams for Thin Lenses}
\begin{outline}
\1 
\end{outline}
\subsection{Optical Instruments}
\begin{outline}
\1 
\end{outline}
\section{Physical Optics}
\subsection{Existence of Light Waves}
\begin{outline}
\1 
\end{outline}
\subsection{Interference}
\begin{outline}
\1 
\end{outline}
\subsection{Young's Double Slit Experiment}
\begin{outline}
\1 
\end{outline}
\subsection{Single Slit Diffraction}
\begin{outline}
\1 
\end{outline}
\subsection{Diffraction by a Circular Aperture}
\begin{outline}
\1 
\end{outline}
\subsection{Resolution}
\begin{outline}
\1 
\end{outline}
\subsection{The Double Slit Revisited}
\begin{outline}
\1 
\end{outline}
\subsection{Multiple Slits: The Diffraction Grating}
\begin{outline}
\1 
\end{outline}
\subsection{Resolution and Angular Dispersion of a Diffraction Grating}
\begin{outline}
\1 
\end{outline}
\subsection{The Index of Refraction and the Speed of Light}
\begin{outline}
\1 
\end{outline}
\subsection{Thin-Film Interference}
\begin{outline}
\1 
\end{outline}
\subsection{Polarized Light}
\begin{outline}
\1 
\end{outline}
\subsection{Polarization by Absorption}
\begin{outline}
\1 
\end{outline}
\subsection{Malus's Law}
\begin{outline}
\1 
\end{outline}
\subsection{Polarization by Reflection: Brewster's Law}
\begin{outline}
\1 
\end{outline}
\subsection{Polarization by Double Refraction}
\begin{outline}
\1 
\end{outline}
\subsection{Polarization by Scattering}
\begin{outline}
\1 
\end{outline}
\end{document}
