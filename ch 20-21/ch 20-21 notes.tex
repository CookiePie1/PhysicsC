\documentclass[twocolumn]{article}
\usepackage[margin=1in]{geometry}
\usepackage{outlines}
\usepackage{amsmath}
\usepackage{gensymb}
\title{Ch. 20-21 Notes}
\author{John Yang}
\setcounter{section}{+19}

\begin{document}
\maketitle
\section{Magnetic Forces and the Magnetic Field}
\subsection{The Magnetic Field}
\begin{outline}
	\1 The end of a compass needle that points generally in a northerly direction at most places on earth is defined to be the North magnetic pole of the compass needle
	\1 The opposite end is defined to be its south pole. 
	\1 Magnets always occur in dipoles. If you break a permanent magnet in half you get two magnets. 
	\1 Like poles repel and unlike poles attract
	\1 Force on a charged particle moving through a uniform field: \[\vec{F}_{\text{magnet on q}}=q\vec{v}\times\vec{B}\]
	\1 Magnetic field $\vec{B}$ is in Teslas (T)
	\1 Use the pointing rhr for positive charges, where the index finger is the velocity, the thumb is the force, and the other fingers are the field. The left hand can be used for negative charges but do not get confused!
\end{outline}
\subsection{Applications}
\begin{outline}
	\1 Velocity selectors; charged particles are shot through perpendicular electric and magnetic fields; only those with a certain velocity make it through without being deflected due to imbalanced forces, given by \(v_0=\dfrac{E}{B}\)
	\1 Mass spectrometers: A charged particle is shot into a region of a uniform magnetic field at a known velocity. The semicircular radius it makes before hitting a plate depends on its mass and is given by \(R=\dfrac{mv}{|q|B}\)
		\2 Can be used to separate a compound into its constituent ions to find its composition. 
	\1 Hall effect: charges moving through a magnetic field experience a force, but moving magnetic fields can also induce a current. Using semiconductors, the Hall effect can be used to measure the proximity of a moving magnet by measuring the way charges move in a semiconductor in response to the change in magnetic field. 

\end{outline}
\subsection{Magnetic Forces on Currents}
\begin{outline}
	\1 Magnetic force on a current-carrying wire: \[\vec{F}_{\text{magnet}}=I\int_{\text{wire}}d\vec{\ell}\times\vec{B}\]

\end{outline}
\subsection{Work Done by Magnetic Forces}
\begin{outline}
	\1 Work done by magnetic forces is always 0 because it is perpendicular to the motion of the charge. 
\end{outline}
\subsection{Torque on a Current Loop in a Magnetic Field}
\begin{outline}
	\1 Magnetic dipole moment: \[\vec{\mu}\equiv I\vec{A}\] where $\vec{A}$ is the area vector. 
	\1 Torque on a current loop inside a uniform magnetic field: \[\vec{\tau}=\vec{\mu}\times\vec{B}=I\vec{A}\times\vec{B}\]
	\1 The current loop inside the magnetic field executes simple harmonic motion. 
\end{outline}
\subsection{The Biot-Savart Law}
\begin{outline}
	\1 To find the magnetic field caused by the entire wire, \[\vec{B}=\dfrac{\mu_0}{4\pi}I\int_{\text{wire}}\dfrac{d\vec{\ell}\times\vec{r}}{r^2}\]
	\1 Some commonly used magnetic fields: 
		\2 At the center of a circular loop of radius $R$ \[\vec{B}_{\text{center}}=\dfrac{\mu_0}{4\pi}\dfrac{2\pi I}{R}\hat{k}\]
		\2 On the axis of a circular current loop \[\vec{B}_{axis}=\dfrac{\mu_0}{4\pi}\dfrac{I(2\pi R)^2}{(R^2+z^2)^{3/2}}\hat{k}\]
		\2 For a circular coil of $n$ loops, all of the same radius, multiply the preceding results by $n$
		\2 A distance $d$ from an infinite wire \[B=\dfrac{\mu_0}{4\pi}\dfrac{2I}{d}\] use grabbing rhr for direction 
		\2 Inside a long solenoid having $n$ turns per meter of its length, each carrying current $I$, far from its ends \[B=\mu_0nI\]
\end{outline}
\subsection{Forces of Parallel Currents on Each Other and the definition of the Ampere}
\begin{outline}
	\1 Forces on parallel wires - attractive if currents in the same direction, repulsive if currents are antiparallel
	\1 Two infinitely long parallel straight current carrying wires exert forces on a length $\ell$ of either wire of magnitude \[F=\dfrac{\mu_0}{4\pi}\dfrac{2I_1I_2}{d}\ell\]

\end{outline}
\subsection{Gauss' Law for the Magnetic Field}
\begin{outline}
	\1 The flux of the magnetic field through any closed surface must always be zero: \[\int_{\text{clsd surf}}\vec{B}\cdot d\vec{S}=0\text{ T}\cdot\text{m}^2\]

\end{outline}
\subsection{Magnetic Poles and current loops}
\begin{outline}
	\1 Magnetic field forms closed loops in accordance with Gauss's law for magnetic field 
	\1 Field lines come out of the north pole and enter into the south pole
\end{outline}
\subsection{Ampere's Law}
\begin{outline}
	\1 Ampere's law is: \[\int_{\text{clsd path}}\vec{B}\cdot d\ell=\mu_0I\]
\end{outline}
\subsection{The Displacement Current and the Ampere-Maxwell Law}
\begin{outline}
	\1 Displacement current - current due to changing electric flux, where \[I_D\equiv\varepsilon_0\dfrac{d\Phi_{\text{elec}}}{dt}\]
	\1 Ampere-Maxwell law - right side of Ampere's law should include both displacement and conduction currents: \[\int_{\text{clsd path}}\vec{B}\cdot d\vec{r}=\mu_0(I+I_D)_{\text{threading the path}}\]
	\1 Makwell realized that magnetic fields are produced by: 
		\2 Electric charges in motion (conduction current)
		\2 Time-varying electric fields (displacement current)
	\1 The magnetic field induced by the displacement current is perpendicular to the (changing) electric field that causes it
	
\end{outline}
\subsection{Magnetic Materials}
\begin{outline}
	\1 certain materials are naturally magnetic. It depends on the relative permeability of the substance: \[\kappa_m=\dfrac{\mu}{\mu_0}\]
		\2 Diamagnetic - $\kappa_m$ slightly less than 1
		\2 Paramagnetic - $\kappa_m$ slightly greater than 1
		\2 Ferromagnetic - $\kappa_m$ much greater than 1
\end{outline}
\subsection{The Magnetic Field of the Earth}
\begin{outline}
	\1 Earth has a permanent magnetic field. 
	\1 Magnetic north pole is actually the geographic south pole
	\1 Magnetic declination - difference in angle between magnetic north and true north
\end{outline}
\section{Faraday's Law of Electromagnetic Induction}
\subsection{Faraday's Law of Electromagnetic Induction}
\begin{outline}
	\1 A changing magnetic flux through a loop induces an electric field, called an induced electric field
	\1 Faraday's law: \[\int_{\text{clsd path}}\vec{E}\cdot d\vec{\ell}=-\dfrac{d\Phi}{dt}\] where $\Phi$ is the magnetic flux through the enclosed area 
	\1 Induced emf: \[\text{induced emf}\equiv\int_{\text{clsd path}}\vec{E}\cdot d\vec{\ell}\]
	\1 Faraday's law is commonly written as \[\text{induced emf}=-\dfrac{d\Phi}{dt}\]
	\1 The magnetic flux is: \[\Phi=\vec{B}\cdot\vec{A}\]
	\1 Changing magnetic fields give rise to induced electric fields via Faraday's law
	\1 Changing electric fields give rise to magnetic fields via the displacement current and the Ampere-Maxwell law.
	\1 the electric and magnetic fields are mutually perpendicular to each other
\end{outline}
\subsection{Lenz's Law}
\begin{outline}
	\1 The induced current always will be directed so as to oppose the change in the magnetic flux that is taking place. 
\end{outline}
\subsection{An ac generator}
\begin{outline}
	\1 A coil of wire of $N$ loops (of identical area $A$) rotates in a uniform magnetic field at an angular frequency $\omega$:\[\text{induced emf}=NBA\omega\sin(\omega t)\]
		\2 The induced emf is sinusoidal 
	\1 When associated with a generator, we call the induced emf the source voltage
\end{outline}
\subsection{Summary of the Maxwell Equations of Electromagnetism}
\begin{outline}
	\1 Gauss' law for electric field: \[\int_{\text{clsd srfc } S}\vec{E}\cdot d\vec{S}=\dfrac{Q_{\text{enc}}}{\varepsilon_0}\]
	\1 Gauss' law for magnetic field: \[\int_{\text{clsd srfc } S}\vec{B}\cdot d\vec{S}=0\text{ T}\cdot\text{m}^2\]
	\1 Ampere-maxwell law: \[\int_{\text{clsd path}}\vec{B}\cdot d\vec{\ell}=\mu_0(I+I_D)=\mu_0I+\mu_0\varepsilon_0\dfrac{d\Phi_{\text{elec}}}{dt}\]
	\1 Faraday's law: \[\int_{\text{clsd path}}\vec{E}\cdot d\vec{\ell}=-\dfrac{d\Phi}{dt}\]
	\1 Light is an electromagnetic wave with its speed given by: \[c=\dfrac{1}{\sqrt{\mu_0\varepsilon_0}}\]
\end{outline}
\subsection{Electromagnetic Waves}
\begin{outline}
	\1 electric field component: \[\dfrac{\partial^2E_y}{\partial x^2}-\mu_0\varepsilon_0\dfrac{\partial^2E_y}{\partial t^2}=0\text{ N}/(\text{C}\cdot\text{m}^2)\]
	\1 Magnetic field component: \[\dfrac{\partial^2B_z}{\partial x^2}-\mu_0\varepsilon_0\dfrac{\partial^2B_z}{\partial t^2}=0\text{ T}/\text{m}^2)\]
	\1 The electric field, magnetic field, and direction of propagation are all mutually orthogonal, therefore light is a transverse wave 
	\1 Acceleration of charges is light
\end{outline}
\subsection{Self-Inductance}
\begin{outline}
	\1 Self inductance is: \[L\equiv\dfrac{\Phi}{I}\] which is the total magnetic flux over the current. Measured in Henrys $H$

\end{outline}
\subsection{Series and Parallel Combinations of Inductors}
\begin{outline}
	\1 Inductors add in series: \[L_{\text{eq}}=L_1+L_2+L_3+\cdots\]
	\1 Inverses of inductors add in parallel: \[\dfrac{1}{L_{\text{eq}}}=\dfrac{1}{L_1}+\dfrac{1}{L_2}+\dfrac{1}{L_3}+\cdots\]
\end{outline}
\subsection{A Series LR Circuit}
\begin{outline}
	\1 For a series LR Circuit, \[I(t)=\dfrac{V_0}{R}\left[1-e^{-(R/L)t}\right]\]
		\2 When I is constant, there is no voltage across the inductor. An Inductor only has an effect while the current is changing with time. For the steady-state situation, there is no potential difference across an inductor since the current no longer is changing with time. \[V=L\dfrac{dI}{dt}\]
	\1 Time constant of a LR circuit is \[\tau\equiv\dfrac{L}{R}\]
\end{outline}
\subsection{Energy stored in a magnetic field}
\begin{outline}
	\1 Potential energy stored by the inductor is: \[U=\dfrac{1}{2}LI^2\]
		\2 That is, the inductor stores energy even if the current is not changing. 
	\1 Power absorbed by the inductor: \[P=\dfrac{dU}{dt}=IL\dfrac{dI}{dt}\]
	\1 Magnetic energy density is \[\dfrac{1}{2}\dfrac{B^2}{\mu_0}\]
\end{outline}
\subsection{A Parallel LC Circuit}
\begin{outline}
	\1 For a parallel LC circuit, \[\dfrac{d^2Q}{dt^2}+\dfrac{1}{LC}Q=0\text{ A}/\text{s}\]
		\2 Which is the equation for SHM
	\1 the charge oscillates between the inductor and the capacitor: \[Q(t)=Q_0\cos(\omega t+\phi)\]
	\1 The angular frequency of the charge oscillations is: \[\omega=\dfrac{1}{\sqrt{LC}}\]
\end{outline}
\subsection{Mutual Inductance}
\begin{outline}
	\1 an inductor near another inductor can induce a current there. 
	\1 mutual inductance \[M_{21}\equiv\dfrac{\Phi_{21}}{I_1}\] \[V_2\equiv M_{21}\dfrac{dI_1}{dt}\] for a closed circuit near an open one \[V_1\equiv M_{21}\dfrac{dI_2}{dt}\] for a reciprocal arrangement \[M_{21}=M_{12}\]
	\1 If both circuits are closed, \[V_1=L_1\dfrac{dI_1}{dt}+M\dfrac{dI_2}{dt}\] \[V_2=L_2\dfrac{dI_2}{dt}+M\dfrac{dI_1}{dt}\]
\end{outline}
\subsection{An Ideal Transformer}
\begin{outline}
	\1 Transformers can change voltage of AC signals using mutual inductance. 
\0 \[V_1=N_1\dfrac{d\Phi}{dt}\] \[V_2=N_2\dfrac{d\Phi}{dt}\] but since $\dfrac{d\Phi}{dt}$ is the same for all wires in the system, we can write this as \[\dfrac{V_1}{V_2}=\dfrac{N_1}{N_2}\]
	\1 Input coil is the primary coil, the output coil is the secondary
	\1 Step up transformers - \(V_2>V_1\) and \(N_2>N_1\)
	\1 Step down transformers - \(V_2<V_1\) and \(N_2<N_1\)
	\1 In a step up transformer, the output potential is greater than the input but the current is less 
	\1 In a step down transformer, the output voltage is less than the input, and the current is greater 
	\1 Transformers don't work with dc current. 
\end{outline}
\end{document}