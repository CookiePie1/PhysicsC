\documentclass[twocolumn]{article}
\usepackage[margin=1in]{geometry}
\usepackage{outlines}
\usepackage{amsmath}
\usepackage{gensymb}
\title{Ch. 16-17 Notes}
\author{John Yang}
\setcounter{section}{+15}

\begin{document}
\maketitle
\section{Electric Charges, Electrical Forces, and the Electric Field}
\subsection{The discovery of electrification}
\begin{outline}
	\1 Electric Force is a central force
	\1 Electric Force is a conservative force
	\1 Glass rubbed with silk is positively charged
	\1 Amber or rubber rubed with fur is negatively charged
	\1 "positive and negative" were arbitrarily chosen
	\1 conductors - conduct electricity
	\1 insulators - don't conduct, also called dielectrics
	\1 charge ($q$ or $Q$) is scalar
	\1 Charge is conser
	\1 similar charges repel and opposite charges attract with equal and opposite forces
\end{outline}
\subsection{Polarization and Induction}
\begin{outline}
	\1 Neutrally charged objects are also attracted by charged objects
	\1 Polarization - charge separation in electrically neutral materials caused by the presence of a nearby electrically charged object
	\1 Charging by Induction
		\2 Take a negatively charged object and hold it fixed near a neutral conductor. The conductor becomes polarized. 
		\2 Ground the conductor to the earth. 
		\2 Remove the grounding wire. The conductor is now positively charged because the electrons have escaped to the earth and cannot return.

\end{outline}
\subsection{Coulomb's Force Law for Pointlike Charges}
\begin{outline}
	\1 Coulomb's law: \[F_{elec}=k\dfrac{|q||Q|}{r^2}\] where \[k=\dfrac{1}{4\pi\epsilon_0}\] where \(k=9.0\times10^9 \text{ N}\cdot\text{m}^2\text{/C}^2\) and \(\epsilon_0=8.85\times10^-12\text{ F/m}\)
	\1 Electric force is a vector
	\1 principle of superposition - total electric force on a changed particle is the vector sum of all the electric forces on the particle
\end{outline}
\subsection{Charge Quantization}
\begin{outline}
	\1 Fundamental unit of charge: \(e=1.602177\times10^{-19}\text{ C}\)
	\1 Charge can only exist in multiples of $e$; that is, \(q=ne\)

\end{outline}
\subsection{The electric field of Static Charges}
\begin{outline}
	\1 Electrical field at a point in space is defined as \[\vec{F_e}\equiv q_{test}\vec{E}\] where $F_e$ is the force on the test charge $q_{test}$ at its location. 
	\1 Electric field is also the force per coulomb
	\1 The test charge is not significant enough to alter the distribution of charge creating the field. 
	\1 the test charge is free to accelerate in response to the field it experiences. 
	\1 The electric field depends on the specific arrangement of the charges creating it. 
\end{outline}
\subsection{The electric field of pointlike charge distributions}
\begin{outline}
	\1 $\vec{E}$ at a distance from a point charge: \[\vec{E}=k\dfrac{Q}{r^2}\hat r\]
\end{outline}
\subsection{A way to visualize electric field: electric field lines}
\begin{outline}
	\1 Electric field line conventions: 
		\2 The electric field lines begin on positive charges and end on negative ones.
		\2 Very close to pointlike charges, the lines are radially symmetric and their number is a measure of the magnitude of the charge. 
		\2 The number of lines passing through a square meter oriented perpendicular to the lines is proportional to the magnitude of the electric field in that region. 
	
\end{outline}
\subsection{A common molecular charge distribution: the electric dipole}
\begin{outline}
	\1 The electric dipole moment is defined as the vector product of the magnitude of either charge times the position vector with respect to the negative charge: \[\vec{p}\equiv|Q|\vec{d}\]
	\1 While the total force on the dipole is zero in a uniform field, the total torque on the dipole depends on the orientation of the dipole moment vector with respect to the direction of the electric field. 

\end{outline}
\subsection{The electric Field of Continuous Distributions of Charge}
\begin{outline}
	\1 The electric field of a continuous distribution of charge can be thus derived: \[\vec{E}=k\int_{chg dist}\dfrac{dq}{r^2}\hat r\]
	\1 The electric field at a point in space depends on: 
		\2 the charge creating the field and its sign 
		\2 the geometric shape of the charge distribution creating the field;
		\2 the distance of the point from the charge distribution
		\2 the placement of the point with respect to the charge distribution (in particular, the location of the point with respect to any symmetry axes associated with the charge distribution)
	\1 Electric field created by various charge distributions: 
		\2 A pointlike charge: \[\vec{E}=k\dfrac{Q}{r^2}\hat r\]
		\2 Dipole along the axis \((z>>d)\) \[\vec{E}=k\dfrac{2p}{z^3}\hat k\] where $z$ is the distance you're measuring from and $d$ is the seperation between the two particles, and $\hat k$ is in the direction towards the positive charge 
		\2 Dipole in the perpendicular bisector plane: \[\vec{E}_{total}=-k\dfrac{p}{\left(y^2+\dfrac{d^2}{4}\right)^{3/2}}\hat k\] where y is the distance you're measuring, perpendicular to the axis. $\hat k$ is still towards the positive charge. For \(y>>d\), \[\vec{E}_{total}=-k\dfrac{p}{y^3}\hat k\]
		\2 At a distance from a uniformly charged ring: \[\vec{E}=k\dfrac{zQ}{(z^2+R^2)^{3/2}}\hat z\] where $R$ is the radius of the ring and $z$ is the distance perpendicular to the ring measured from the center. 
		\2 At a distance from a uniformly changed disk: \[\vec{E}=k\dfrac{2Q}{R^2}\left[1-\dfrac{z}{\sqrt{z^2+R^2}}\right]\hat z\]
		\2 A uniformly charged infinite sheet: \[E=\dfrac{\sigma}{2\epsilon_0}\] (not vector), $\sigma$ is the charge per unit area 
		\2 Two infinite uniformly charged sheets with opposite charge: \[E=\dfrac{\sigma}{\epsilon_0}\]
		\2 A uniformly charged spherical shell: Inside \((r<R)\): \(\vec{E}=0\text{ N/C}\), outside \((r>R)\): \[\vec{E}=k\dfrac{Q}{r^2}\hat r\]
		\2 Inside a uniformly charged sphere \((r<R)\): \[\vec{E}=k\dfrac{Q}{R^3}r\hat r\]
		\2 Outside a uniformly charged sphere \((r>R)\): \[\vec{E}=k\dfrac{Q}{r^2}\hat r\]
		\2 An infinitely long charge in a straight line: \[\vec{E}=k\dfrac{2\lambda}{r}\hat r\], where $\lambda$ is the charge per unit length and $r$ is measured raidally away from the line. 

\end{outline}
\subsection{Motion of a charged particle in a uniform electric field: an electrical projectile}
\begin{outline}
	\1 Experiences a constant force in both magnitude and direction because the field is uniform.
\end{outline}
\subsection{Gauss' law for electric fields}
\begin{outline}
	\1 flux $\Phi$ is: \[d\Phi=\vec{A}\cdot d\vec{S}\]
	\1 Gauss' law for E fields: \[\int_{\text{clsd srfc}}\vec{E}\cdot d\vec{S}=\dfrac{Q_{\text{enclosed}}}{\epsilon_0}\], where you integrate over the closed surface $\vec{S}$, and $Q_{\text{enclosed}}$ is the charge enclosed by the surface $\vec{S}$
	\1 flux is the flux of the field caused by all charges everywhere, even those not enclosed by the gaussian surface. 
\end{outline}
\subsection{Calculating the Magnitude of the electric field using Gauss' law}
\begin{outline}
	\1 Gauss' law is general. That is, the closed surface is arbitrary and you can choose a surface that can help you find the magnitude of a field in question. 
\end{outline}
\subsection{Conductors}
\begin{outline}
	\1 Charge separation in a conductor continues until the total field inside the conductor is zero
	\1 Therefore, no free charge can exist anywhere within the conductor.
	\1 That means that any free charge must be on the surface of the conductor. 
	\1 The electric field vector at the surface of a conductor must be perpendicular to the surface. 
	\1 The magnitude of the field at the surface is given by \[E\dfrac{\sigma}{\epsilon_0}\] where $\sigma$ is the surface charge density
\end{outline}
\subsection{Other electrical materials}
\begin{outline}
	\1 semiconductors - not good conductors nor good insulators 
	\1 superconducors - lose all electrical resistance
\end{outline}
\section{Electric Potential Energy and the Electric Potential}
\subsection{Electric potential energy and the electric potential}
\begin{outline}
	\1 Electric force is a conservative force
	\1 Electric potential is defined as the potential energy per unit charge at a point in space: \[V\equiv\dfrac{PE_{\text{of q}}}{q}\] Units: volt (V$=$J/C)
	\1 This means that \[PE=qV\]
	\1 Electric potential and electric PE are scalars.
	\1 Potential difference or change in potential b/w two points: \[\Delta V=-\int^f_i\vec{E}\cdot d\vec{r}\]
	\1 An electrical ground is any location where the potential is zero. You can choose where the ground is. 
\end{outline}
\subsection{The electric potential of a pointlike charge}
\begin{outline}
	\1 For a point charge, you can choose the location of ground to be infinity (out of convenience/convention)
	\1 For a point charge, \[V(r)=k\dfrac{Q}{r}\]
\end{outline}
\subsection{the electric potential of a collection of pointlike charges}
\begin{outline}
	\1 Electric potential is additive
\end{outline}
\subsection{The electric potential of continuous charge distributions of finite size}
\begin{outline}
	\1 For a charge distribution of finite size, \[V=k\int_{\text{finite chg dst}}\dfrac{dQ}{r}\]
	\1 The electric potential of various charge distributions: 
		\2 Two infinite oppositely charged sheets (uniform field): \[V(x)-V(0)=-Ex\]
		\2 A pointlike charge: \[V(r)=k\dfrac{Q}{r}\]
		\2 A uniformly charged ring at a distance $z$ from the center along its axis of symmetry: \[V=k\dfrac{Q}{\sqrt{\left(z^2+R^2\right)}}\]
		\2 A uniformly charged disk at a distance $z$ from the center along its axis: \[V=k\dfrac{2Q}{R^2}\left[\sqrt{\left(z^2+R^2\right)}-z\right]\]
		\2 Outside a uniformly charged sphere \((r>R)\): \[V(r)=k\dfrac{Q}{r}\]
		\2 Inside a uniformly charged conducting sphere \((r<R)\): constant \[V=k\dfrac{Q}{R}\]
		\2 Inside a uniformly charged insulating sphere \((r<R)\): \[V(r)=k\dfrac{Q}{2R}\left(3-\dfrac{r^2}{R^2}\right)\]

\end{outline}
\subsection{equipotential volumes and surfaces}
\begin{outline}
	\1 Equipotentials: volumes, surfaces, lines that have the same potential, that is, \(\Delta V=0\)
	\2 In a uniform field, equipotentials are planes
	\2 Outside a pointlike charge, equipotentials are spheres centered at the charge 
\end{outline}
\subsection{The relationship between the electric potential and the electric field}
\begin{outline}
	\1 Electric field is the negative derivative of potential with respect to $s$, that is \[E_s=-\dfrac{dV}{ds}\]
	\1 More generally, electric field is the negative gradient of the potential: \[\vec{E}=-\nabla V\]
\end{outline}
\subsection{Acceleration of charged particles under the influence of electrical forces}
\begin{outline}
	\1 Energy is still conserved. 
	
\end{outline}
\subsection{A new unit of energy: the electron-volt}
\begin{outline}
	\0 \[1\text{ eV}=1.602\times10^{-19}\text{ J}\]
\end{outline}
\subsection{An electric dipole in an external electric field revisited}
\begin{outline}
	\1 Potential energy of the dipole is \[PE_{\text{dipole}}=-\vec{p}\cdot\vec{E}\]
\end{outline}
\subsection{The electric potential and electric field of a dipole}
\begin{outline}
	\1 If $r$ is close to the dipole (similar to $d$), then \[V_{\text{dipole}}=k\dfrac{p\cos\theta}{r^2}\]
	\1 If $r$ is much greater than $d$, then \[E_x=-\dfrac{\partial V}{\partial x}=k\dfrac{3zxp}{(x^2+y^2+z^2)^{5/2}}\] \[E_y=-\dfrac{\partial V}{\partial y}=k\dfrac{3zyp}{(x^2+y^2+z^2)^{5/2}}\] \[E_z=-\dfrac{\partial V}{\partial z}=-kp\left[\dfrac{1}{(x^2+y^2+z^2)^{3/2}}-\dfrac{3z^2}{(x^2+y^2+z^2)^{5/2}}\right]\]
\end{outline}
\subsection{The potential energy of a distribution of pointlike charges}
\begin{outline}
	\1 Potential energy and potential are additive
	\1 with no external forces, \[W_{\text{elec}}=-PE_{\text{total f}}\]
\end{outline}
\subsection{lightning rods}
\begin{outline}
	\1 Consider a small sphere connected to a larger sphere by a long wire. The magnitude of the electric field at the smaller shpere is greater than that at the larger sphere. 
	\1 Lightning rods have very sharp tips. This means that the tip can easily reach the breakdown voltage of air meaning the lightning strikes the lightning rod and not a tree or a building. 
\end{outline}
\end{document}
